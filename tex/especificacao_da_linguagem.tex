\section{Especificação da linguagem}
\label{sec:estrutura-geral}

\subsection{Estrutura geral}
\label{subsec:estrutura-geral}
Um programa escrito na linguagem Agoravai precisa conter uma função chamada \texttt{main} que recebe um
\textit{array} de \textbf{strings} (argumentos da linha de comando) e retorna um código de erro (0 caso o programa
seja executado com sucesso).
Comentários iniciam com \texttt{//} e o restante da linha em que aparecem é ignorado.
Um bloco de código é uma região delimitada por \texttt{\{} e \texttt{\}}.
Variáveis podem ser declaradas globalmente (fora de qualquer função/bloco) ou localmente (dentro de uma função ou
bloco).
O ambiente de referenciamento da variável é o bloco a qual ela pertence no caso de váriaveis locais, e qualquer
ponto do programa para as globais.
Para referenciar qualquer variável em um determinado ponto do programa é necessário que essa variável tenha sido
declarada em um ponto anterior.

\subsubsection{Nomes}
A tabela\ref{tab:palavras-reservadas} mostra as palavras reservadas da linguagem.
Identificadores de variáveis e funções devem iniciar com uma letra ou \texttt{\textunderscore} e podem conter
letras, dígitos ou \texttt{\textunderscore}, sem limite de caracteres e obviamente não podem estar contidos no
conjunto das palavras reservadas.
\begin{table}[]
    \centering
    \begin{tabular}{ccccc}
        {bool} & 
{break} & 
{char} & 
{if} & 
{else} & 
{false} & 
{float} & 
{for} & 
{int} & 
{in} & 
{print} & 
{return} & 
{scan} & 
{skip} & 
{string} & 
{true} & 
{void} & 
{while}
    \end{tabular}
    \caption{Palavras reservadas}
    \label{tab:palavras-reservadas}
\end{table}

\subsubsection{Declarações}
\label{sec:declaracoes}

\paragraph{Variáveis} Para declarar uma variável é necessário especificar o tipo e o nome dela.
Opcionalmente, é possível inicializá-la com uma expressão cujo resultado seja do mesmo tipo da variável ou pode
sofrer uma coerção para ele.
Também é possível declarar mais de uma variável do mesmo tipo em uma única sentença, separando-as por vírgula.
Ex.: \texttt{\textbf{int} a;}, \texttt{\textbf{int} a = 1}, \texttt{\textbf{int} b = 2, c = 4;} e
\texttt{\textbf{int} d = b + c}.

\paragraph{Subprogramas} Subprogramas podem ser procedimentos ou funções.
A sintaxe de declaração de um subprograma é mostrada na listagem \ref{lst:subprogramas}
O retorno da função deve ser uma expressão compatível com o tipo declarado na assinatura da função.
Para mais informações sobre subprogramas ver seção \ref{sec:funcoes}.
\begin{lstlisting}[language=C, caption=Declaração de subprogramas, label=lst:subprogramas]
        // Procedimento
        void my_procedure(int a) {
            ...
        }

        // Funcao
        int my_func(int a) {
            ...
            return (int) ...;
        }
\end{lstlisting}
\subsection{Especificação de tipos}

\subsubsection{bool}
Representa valores booleanos.

\paragraph{Constantes}
Os únicos valores possíveis são \texttt{true} (verdadeiro) e \texttt{false} (falso).

\paragraph{Operadores} A tabela \ref{tab:operadores-bool} mostra os operadores para valores do tipo \textbf{bool}. Considere que \texttt{a} e \texttt{b} são duas variáveis \textbf{bool}.
\begin{table}[!h]
    \centering
    \begin{tabular}{ l c c l }
        Operador      & Associatividade & Precedência & Descrição                                                        \\
        \hline
        \texttt{==}   & Esquerda        & 1           & \texttt{a} é igual a \texttt{b}?                                 \\
        \texttt{!=}   & Esquerda        & 1           & \texttt{a} é diferente de \texttt{b}?                            \\
        \texttt{\&\&} & Esquerda        & 2           & \texttt{a} e \texttt{b} são verdadeiros?                         \\
        \texttt{||}   & Esquerda        & 2           & \texttt{a} ou \texttt{b} são verdadeiros?                        \\
        \texttt{!}    & Direita         & 3           & o inverso de \texttt{a}                                          \\
        \texttt{++}   & Esquerda        & 2           & concatena a \textbf{string} que representa o valor de \texttt{a} \\
        \hline
    \end{tabular}
    \caption{Operadores do tipo \textbf{bool}}
    \label{tab:operadores-bool}
\end{table}

\paragraph{Coerções} A tabela \ref{tab:coercoes-bool} mostra as possíveis coerções do tipo \textbf{bool}.
\begin{table}[!h]
    \begin{tabular}{@{}ll@{}}
        \toprule
        Tipo            & Descrição                                                                         \\ \midrule
        \textbf{int}    & Retorna \texttt{1} para \texttt{true} e \texttt{0} para \texttt{false}            \\
        \textbf{string} & Retorna \texttt{"true"} para \texttt{true} e \texttt{"false"} para \texttt{false} \\ \bottomrule
    \end{tabular}
    \caption{Coerções do tipo bool}
    \label{tab:coercoes-bool}
\end{table}

\subsubsection{char}
Representa um caractere. O caractere deve pertencer à tabela ASCII. Caracteres de controle são representados por uma sequência de escape (ver tabela \ref{char_control_seq})

\paragraph{Constantes} Um caractere ou uma sequência de escape entre \texttt{'} (apóstrofos). Ex.: \texttt{'a'}, \texttt{'0'}, \texttt{'\\n'}.

\paragraph{Operadores}  A tabela \ref{tab:operadores-char} mostra os operadores para valores do tipo \textbf{char}. Considere que \texttt{a} e \texttt{b} são duas variáveis \textbf{char}.
\begin{table}[!h]
    \begin{tabular}{@{}llll@{}}
        \toprule
        Operador                 & Associatividade & Precedência & Descrição                                                    \\
        \midrule
        \texttt{==}              & Esquerda        & 1           & \texttt{a} é igual a \texttt{b}?                                                  \\
        \texttt{!=}              & Esquerda        & 1           & \texttt{a} é diferente de \texttt{b}?                                             \\
        \texttt{\textless{}}     & Esquerda        & 1           & \texttt{a} é menor que \texttt{b}?                                   \\
        \texttt{\textless{}=}    & Esquerda        & 1           & \texttt{a} é menor ou igual a \texttt{b}?                            \\
        \texttt{\textgreater{}}  & Esquerda        & 1           & \texttt{a} é maior que \texttt{b}?                                   \\
        \texttt{\textgreater{}=} & Esquerda        & 1           & \texttt{a} é maior ou igual a \texttt{b}?                            \\
        \texttt{++}              & Esquerda        & 2           & \textbf{string} resultado da concatenação de \texttt{a} e \texttt{b}
    \end{tabular}
    \caption{Operadores do tipo \textbf{char}}
    \label{tab:operadores-char}
\end{table}

\paragraph{Coerções} A tabela \ref{tab:coercoes-char} mostra as possíveis coerções do tipo \textbf{char}.
\begin{table}[!h]
    \begin{tabular}{@{}ll@{}}
        \toprule
        Tipo            & Descrição                                                      \\ \midrule
        \textbf{int}    & Retorna o valor do caractere na tabela ASCII                   \\
        \textbf{string} & Retorna uma \textbf{string} composta unicamente pelo caractere \\ \bottomrule
    \end{tabular}
    \caption{Coerções do tipo \textbf{char}}
    \label{tab:coercoes-char}
\end{table}

\subsubsection{int}
Representa números inteiros.

\paragraph{Constantes} Sequência de dígitos (0 a 9), sem zeros à esquerda. Podem conter sinal ou não.

\paragraph{Operadores} A tabela \ref{tab:operadores-int} mostra os operadores para valores do tipo \textbf{int}. Considere que \texttt{a} e \texttt{b} são duas variáveis \textbf{int}.
\begin{table}[!h]
    \begin{tabular}{@{}llll@{}}
        \toprule
        Operador                 & Associatividade & Precedência & Descrição                                                         \\
        \midrule
        \texttt{==}              & Esquerda        & 1           & \texttt{a} é igual a \texttt{b}?                                  \\
        \texttt{!=}              & Esquerda        & 1           & \texttt{a} é diferente de \texttt{b}?                             \\
        \texttt{\textless{}}     & Esquerda        & 1           & \texttt{a} é menor que \texttt{b}?                                \\
        \texttt{\textless{}=}    & Esquerda        & 1           & \texttt{a} é menor ou igual a \texttt{b}?                         \\
        \texttt{\textgreater{}}  & Esquerda        & 1           & \texttt{a} é maior que \texttt{b}?                                \\
        \texttt{\textgreater{}=} & Esquerda        & 1           & \texttt{a} é maior ou igual a \texttt{b}?                         \\
        \texttt{+}               & Esquerda        & 2           & soma de \texttt{a} e \texttt{b}                                   \\
        \texttt{-}               & Esquerda        & 2           & subtração de \texttt{a} por \texttt{b}                            \\
        \texttt{$*$}             & Esquerda        & 3           & produto de \texttt{a} e \texttt{b}                                \\
        \texttt{/}               & Esquerda        & 3           & divisão de \texttt{a} por \texttt{b}                              \\
        \texttt{\%}              & Esquerda        & 4           & resto da divisão de \texttt{a} por \texttt{b}                    \\
        \texttt{++}              & Esquerda        & 2           & concatena a \textbf{string} que representa o valor de \texttt{a}  \\
        \texttt{+} (unário)      & Direita         & 4           & não faz nada                                                      \\
        \texttt{-} (unário)      & Direita         & 4           & oposto de \texttt{a}                                              \\
    \end{tabular}
    \caption{Operadores do tipo \textbf{int}}
    \label{tab:operadores-int}
\end{table}

\paragraph{Coerções} A tabela \ref{tab:coercoes-int} mostra as possíveis coerções do tipo \textbf{int}.
\begin{table}[!h]
    \begin{tabular}{@{}ll@{}}
        \toprule
        Tipo            & Descrição                                                                  \\ \midrule
        \textbf{bool}   & Retorna \texttt{true} para números diferentes de 0 e \texttt{false} para 0 \\
        \textbf{float}  & Retorna o número de ponto flutuante mais próximo                           \\
        \textbf{string} & Retorna a \textbf{string} com a representação do número                    \\ \bottomrule
    \end{tabular}
    \caption{Coerções do tipo \textbf{int}}
    \label{tab:coercoes-int}
\end{table}

\subsubsection{float}
Representa números reais.

\paragraph{Constantes} Sequência de dígitos (0 a 9) representando a parte inteira, seguida de \texttt{.} e por fim uma sequência de dígitos representando a parte decimal

\paragraph{Operadores} A tabela \ref{tab:operadores-float} mostra os operadores para valores do tipo \textbf{float}. Considere que \texttt{a} e \texttt{b} são duas variáveis \textbf{float}.
\begin{table}[!h]
    \begin{tabular}{@{}llll@{}}
        \toprule
        Operador                 & Associatividade & Precedência & Descrição                                 \\ \midrule
        \texttt{==}              & Esquerda        & 1           & \texttt{a} é igual a \texttt{b}?          \\
        \texttt{!=}              & Esquerda        & 1           & \texttt{a} é diferente de \texttt{b}?     \\
        \texttt{\textless{}}     & Esquerda        & 1           & \texttt{a} é menor que \texttt{b}?        \\
        \texttt{\textless{}=}    & Esquerda        & 1           & \texttt{a} é menor ou igual a \texttt{b}? \\
        \texttt{\textgreater{}}  & Esquerda        & 1           & \texttt{a} é maior que \texttt{b}?        \\
        \texttt{\textgreater{}=} & Esquerda        & 1           & \texttt{a} é maior ou igual a \texttt{b}? \\
        \texttt{+}               & Esquerda        & 2           & soma de \texttt{a} e \texttt{b}           \\
        \texttt{-}               & Esquerda        & 2           & subtração de \texttt{a} por \texttt{b}    \\
        \texttt{$*$}             & Esquerda        & 3           & produto de \texttt{a} e \texttt{b}        \\
        \texttt{/}               & Esquerda        & 3           & divisão de \texttt{a} por \texttt{b}      \\
        \texttt{++}              & Esquerda        & 2           & concatena a \textbf{string} que representa o valor de \texttt{a}\\
        \texttt{+} (unário)      & Direita         & 4           & não faz nada                                                      \\
        \texttt{-} (unário)      & Direita         & 4           & oposto de \texttt{a}
    \end{tabular}
    \caption{Operadores do tipo \textbf{float}}
    \label{tab:operadores-float}
\end{table}

\paragraph{Coerções} A tabela \ref{tab:coercoes-float} mostra as possíveis coerções do tipo \textbf{float}.
\begin{table}[!h]
    \begin{tabular}{@{}ll@{}}
        \toprule
        Tipo            & Descrição                                               \\ \midrule
        \textbf{string} & Retorna a \textbf{string} com a representação do número \\ \bottomrule
    \end{tabular}
    \caption{Coerções do tipo \textbf{float}}
    \label{tab:coercoes-float}
\end{table}

\subsubsection{string}
Representa uma cadeia de caracteres.

\paragraph{Constantes}

\paragraph{Operadores} A tabela \ref{tab:operadores-string} mostra os operadores para valores do tipo \textbf{string}. Considere que \texttt{a} e \texttt{b} são duas variáveis \textbf{string}.
\begin{table}[!h]
    \begin{tabular}{@{}llll@{}}
        \toprule
        Operador                 & Associatividade & Precedência & Descrição                                      \\ \midrule
        \texttt{==}              & Esquerda        & 1           & \texttt{a} é igual a \texttt{b}?               \\
        \texttt{!=}              & Esquerda        & 1           & \texttt{a} é diferente de \texttt{b}?          \\
        \texttt{\textless{}}     & Esquerda        & 1           & \texttt{a} vem antes de \texttt{b}?            \\
        \texttt{\textless{}=}    & Esquerda        & 1           & \texttt{a} vem antes ou é igual a \texttt{b}?  \\
        \texttt{\textgreater{}}  & Esquerda        & 1           & \texttt{a} vem depois de \texttt{b}?           \\
        \texttt{\textgreater{}=} & Esquerda        & 1           & \texttt{a} vem depois ou é igual a \texttt{b}? \\
        \texttt{++}              & Esquerda        & 2           & concatenação de \texttt{a} e \texttt{b}        \\
    \end{tabular}
    \caption{Operadores do tipo \textbf{string}}
    \label{tab:operadores-string}
\end{table}

\paragraph{Coerções} A tabela \ref{tab:coercoes-string} mostra as possíveis coerções do tipo \textbf{string}.
\begin{table}[!h]
    \begin{tabular}{@{}ll@{}}
        \toprule
        Tipo            & Descrição                                                                                         \\ \midrule
        \textbf{char[]} & Retorna um \textit{array} de \textbf{char} onde os elementos são os caracteres da \textbf{string} \\ \bottomrule
    \end{tabular}
    \caption{Coerções do tipo \textbf{string}}
    \label{tab:coercoes-string}
\end{table}

\subsubsection{Arrays}
Representa uma coleção homôgenea de valores com tamanho fixo. A declaração de um array é feita de um tipo seguido de \texttt{[]}. Ex.: \textbf{string[]}, \textbf{int[]} etc. O tamanho pode ser explícito ou inferido a partir do valor inicial. Se for explícito só pode ser iniciado com um array de tamanho igual ou menor. Caso seja omitido, o array deverá ser inicializado e terá o tamanho do valor inicial.

\paragraph{Constantes} Um array do tipo T é escrito como uma sequência de elementos do tipo T, separados por vírgula e entre colchetes. Ex.: \texttt{[0, 1, 2, 3]}, \texttt{['a', 'b', 'c', 'd', 'e']}. Também é possível criar um array com a sintaxe \texttt{[<inicio> ... <fim>]}, que cria um array com os elementos de inicio até fim (aberto) onde início e fim devem ser valores de tipos numéricos ou caracteres (coerção para \textbf{int}).

\paragraph{Operadores} A tabela \ref{tab:operadores-array} mostra os operadores para \textit{arrays}. Considere que \texttt{a} e \texttt{b} são dois \textit{arrays}.
\begin{table}[!h]
    \begin{tabular}{@{}llll@{}}
        \toprule
        Operador    & Associatividade & Precedência & Descrição                               \\ \midrule
        \texttt{==} & Esquerda        & 1           & \texttt{a} é igual a \texttt{b}?        \\
        \texttt{!=} & Esquerda        & 1           & \texttt{a} é diferente de \texttt{b}?   \\
        \texttt{+}  & Esquerda        & 2           & concatenação de \texttt{a} e \texttt{b} \\
    \end{tabular}
    \caption{Operadores de \textit{arrays}}
    \label{tab:operadores-array}
\end{table}

\paragraph{Coerções} A tabela \ref{tab:coercoes-array} mostra as possíveis coerções de \textit{arrays}.
\begin{table}[!h]
    \begin{tabular}{@{}ll@{}}
        \toprule
        Tipo            & Descrição                                                                                                                    \\ \midrule
        \textbf{string} & Retorna uma \textbf{string} com a representação em \textbf{string} de cada elemento do \textit{array}, separados por vírgula \\ \bottomrule
    \end{tabular}
    \caption{Coerções de \textit{arrays}}
    \label{tab:coercoes-array}
\end{table}

Para \textit{arrays} e \textbf{strings}, é possível acessar/alterar o $i$-ésimo elemento através de \texttt{[<indice>]}.
\subsection{Instruções}

\subsubsection{Estrutura condicional}
Estruturas condicionais são sentenças que realizam desvios no fluxo do programa dependendo do valor de uma expressão
booleana.
A seguir são apresentadas as estruturas condicionais de uma via e de duas vias.

\paragraph{De uma via}
A listagem \ref{lst:condicional-1-via} mostra a estrutura condicional de uma via.
Se \texttt{expressao} for \texttt{true}, o fluxo do programa segue para a sentença ou bloco após o \texttt{)}.
Caso contrário, o fluxo é desviado para depois da sentença ou bloco seguinte.

\begin{lstlisting}[language=C, caption=Estrutura condicional de uma via, label=lst:condicional-1-via]
    if (<expressao)
        <sentenca>
    if (<expressao>) {
        <lista de sentencas>
    }
\end{lstlisting}

\paragraph{De duas vias}
A listagem \ref{lst:condicional-2-vias} mostra a estrutura condicional de duas vias.
A diferença para a estrutura de uma via é a palavra \texttt{else} após o bloco ou sentença da 1ª via.
De maneira semelhante, o bloco ou sentença que aparece diretamente após o \texttt{else} só será executado(a) se a
expressão da 1ª via for \texttt{false}.

\begin{lstlisting}[language=C, caption=Estrutura condicional de uma via, label=lst:condicional-2-vias]
    if (<expressão>) {
        ...
    }
    else if (<expressão>) { // Como a sentenca if vem depois do else, o else mais abaixo corresponde a esse if
        ...
    }
    else {
        ...
    }
\end{lstlisting}

\subsubsection{Estrutura iterativa}
As estruturas iterativas repetem um bloco ou sentença. Há dois tipos de desvio incondicional dentro de uma estrutura dessa.
O comando \texttt{\textbf{break}} desvia para depois do laço mais próximo e o comando \texttt{\textbf{skip}} desvia do
ponto atual no bloco para o teste e consequentemente a próxima iteração. Nos dois casos mostrados a seguir a avaliação
dos laços é pré-teste.

\paragraph{com controle lógico}
\begin{lstlisting}[language=C, caption=Estrutura iterativa com controle lógico, label=lst:iterativa-logico]
    while(<expressão>) {
        ...
    }
\end{lstlisting}

\paragraph{por contador}
A iteração por contador é mostrada na listagem \ref{lst:iteracao-contador}.
Com ela é possível iterar sobre intervalos (\textit{strings}, \textit{arrays} e intervalos numéricos).
\texttt{tipo do contador} representa o tipo de cada elemento do conjunto (\texttt{char} para \textit{strings} e
tipo correspondente do \textit{array} ou intervalo numérico) e \texttt{nome do contador} representa o nome que
poderá ser usado pelo programador para se referir ao contador.
Para declarar um intervalo numérico é necessário no mínimo uma expressão do tipo correspondente.
Nesse caso entende-se que o valor inicial é 0 e o incremento igual a 1.
A sintaxe utilizada é demonstrada na listagem\ref{lst:iterativa-contador}.
Também é possível especificar um valor inicial e um valor de incremento.
O valor de incremento é deduzido a partir da diferença entre o primeiro e segundo valores.
Caso o segundo valor seja omitido, pressupõe-se que o valor de incremento é 1.
Os tipos permitidos nesse tipo de construção são: \texttt{char}, \texttt{int} e \texttt{float}.

\begin{lstlisting}[language=C, caption=Estrutura iterativa por contador, label=lst:iterativa-contador]
    for(<tipo do contador> <nome do contador> in <intervalo>)
        <sentenca>

    for(<tipo do contador> <nome do contador> in <intervalo>) {
        <lista de sentencas>
    }

    // Ex.:
    for(int i in [1, 2, 3]) {
        ...
    }

    for(char c in "Ola mundo") {
        ...
    }

    // Notacao de intervalo para tipos numericos
    // [<inicio>, ..., <fim>) de <inicio> ate <fim> - 1
    // [<inicio>, <segundo>, ..., <fim>) de inicio ate fim - 1, de g em g, onde g eh igual a (<segundo> - <inicio>)
    for (int i in [1, ..., 20)) { // 1, 2, 3, ..., 18 e 19
        ...
    }
    for (int i in [1, 5, ..., 20] { // 1, 5, 10
        ...
    }
\end{lstlisting}

\subsubsection{Entrada}
A entrada é feita através do comando \texttt{\textbf{scan}}. Ele recebe uma lista com as variáveis que receberão os
valores lidos, na ordem em que são passados. Opcionalmente é possível passar uma \textbf{string} como formato para ser
lido imediatamente à esquerda.

\begin{lstlisting}[language=C, caption=Comando \texttt{scan}, label=lst:entrada]
        int a;
        string s;
        
        scan(a, "%[^\n]" s);
\end{lstlisting}

\subsubsection{Saída}A saída é feita através do comando \texttt{\textbf{print}}.
Ele recebe uma lista com as expressões que serão impressas, na ordem em que são passadas.
Opcionalmente é possível passar uma \textbf{string} como formato para ser impresso imediatamente à esquerda.

\begin{lstlisting}[language=C, caption=Comando \texttt{print}, lst:saida]
        int a = 10;
        string s = "agora vai";
        
        print(a, "%20" s);
\end{lstlisting}

\subsubsection{Atribuição}
A atribuição (\texttt{=}) é um operador com precedência 0 e associatividade à direita. O operando à esquerda deve ser o
identificador de uma variável e à direita uma expressão que resulte num tipo compatível com o da variável.
\subsection{Subprogramas}
\label{sec:funcoes}
É possível declarar subprogramas localmente ou globalmente.
A passagem de parâmetros para subprogramas é feita em modo de entrada. 
Os parâmetros são passados por valor.
A sintaxe para declaração de subprogramas foi mostrada na seção\ref{sec:declaracoes}.
Procedimentos são declarados da mesma forma que funções, exceto que no lugar do tipo de retorno possuem a palavra \texttt{void}.
\subsection{Exemplos de programas}
        \subsubsection{Olá mundo}
        \lstinputlisting[]{exemplos_programas/ola_mundo.agrvai}
        \subsubsection{Série de Fibonacci}
        \lstinputlisting[]{exemplos_programas/serie_fibonacci.agrvai}
        \subsubsection{Shell sort}
        \lstinputlisting[]{exemplos_programas/shell_sort.agrvai}